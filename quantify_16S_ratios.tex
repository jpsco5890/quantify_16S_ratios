\documentclass[twocolumn]{article}
\usepackage{cite}
\usepackage{authblk}
\usepackage{hyperref}
\hypersetup{colorlinks=true, citecolor=black}

\begin{document}

\title{Protocol: Quantifying species abundance using 16S Sanger Sequencing and CASEU}
\author{Jack Reddan}

\maketitle{}

\tableofcontents{}

\section{Culture Collection}\label{CC}
\subsection{Overview}
This section describes collecting cultures for which you wish to quantify species abundances.
In addition to any experimental mixtures,
pure cultures of community members are needed for the analysis.

\subsection{Materials}
\begin{itemize}
  \item Pure and mixed liquid cultures
  \item 2-mL microcentrifuge tubes
\end{itemize}

\subsection{Method}
\begin{enumerate}
  \item In a 2-mL microcentrifuge tube, Collect 1 ODmL of liquid culture for all mixtures and pure cultures. Spin-down and remove supernatant.
  \item Either store pellets at $-20^\circ$C, or continue to \hyperref[GI]{gDNA Isolation}.
\end{enumerate}

\section{gDNA Isolation}\label{GI}
\subsection{Overview}
This section describes how to isolate gDNA to be used for 16S rDNA amplification.
These steps are for the QIAGEN DNeasy Blood \& Tissue Kit.
If you happen use a different kit,
please follow its instructions,
and then proceed to \hyperref[1rA]{16S rDNA Amplification}.

\subsection{Materials}
\begin{itemize}
  \item QIAGEN DNeasy Blood \& Tissue kit
  \item 100\% ethanol
  \item Cell pellets from cultures collected in \hyperref[CC]{Culture Collection}
\end{itemize}

\subsection{Method}
Follow the subsequent instruction in parallel for all samples to be processed.
\begin{enumerate}
  \item Resuspend pellet in 180 $\mu$L of buffer ATL, and add 20 $\mu$L of Proteinase K. Mix thoroughly by vortexing. Incubate at $56^\circ$C for 1 hour, and vortex every 15 minutes.
  \item Vortex for 15s, then add 200 $\mu$L of buffer AL and mix thoroughly by vortexing. Immediately add 200 $\mu$L of 100\% ethanol and, again, mix thoroughly by vortexing.
  \item Pipet the mixture from step 2 into a DNeasy Mini spin column placed in a 2-mL collection tube. Centrifuge at $\geq 6,000$ x $g$ for 1 min. Afterwards, discard flow-through and collection tube.
  \item Place DNeasy Mini spin column into a new 2-mL collection tube and add 500 $\mu$L of buffer AW1. Centrifuge at $\geq 6,000$ x $g$ for 1 min. Afterwards, discard flow-through and collection tube.
  \item Place DNeasy Mini spin column into a new 2-mL collection tube and add 500 $\mu$L of buffer AW2. Centrifuge at $20,000$ x $g$ for 3 min. Afterwards, discard flow-through and collection tube.
  \item Place DNeasy Mini spin column into a clean 1.5-mL microcentrifuge tube. Add 200 $\mu$L of buffer AE and incubate at room temperature for 1 min. Centrifuge at $\geq 6,000$ x $g$ for 1 min to elute.
  \item Check gDNA concentration on the nano spectrophotometer. Then store at $-20^\circ$C or continue to \hyperref[1rA]{16S rDNA Amplification}.
\end{enumerate}

\section{16S rDNA Amplification}\label{1rA}
\subsection{Overview}
This section describes how to amplify the 16S rDNA using Phusion polymerase for Sanger sequencing.
If you are using a different polymerase,
please follow its specific instructions and proceed to \hyperref[SS]{Sanger Sequencing}.

\subsection{Materials}
\begin{itemize}
  \item Phusion polymerase
  \item 5X High Fidelity (HF) reaction buffer
  \item dNTPs
  \item 27F forward and 1492R reverse primers
  \item gDNA template from \hyperref[GI]{gDNA Isolation}
\end{itemize}

\subsection{Method}
Follow the subsequent instruction in parallel for all gDNA samples collected in \hyperref[GI]{gDNA Isolation}.
\begin{enumerate}
  \item Mix the ddH$_2$O, dNTPs, forward\slash reverse primers, 5X HF buffer, and polymerase according to \hyperref[aA]{Appendix A} for $N+1$ reactions, where $N$ is the number of gDNA samples to be processed.
  \item Pipet 38 $\mu$L of the PCR master mix into 200-$\mu$L PCR tubes. Then add 2 $\mu$L of template gDNA, at a concentration of 100 ng\slash $\mu$L, to the mix for each sample.
  \item Place reactions into the thermocycler and run the PCR program specified in \hyperref[aA]{Appendix A}.
  \item Once the run is complete, check for synthesis of the desired product using gel electrophoresis. If the proper product was amplified, either store the reaction at $-20^\circ$ C or proceed to \hyperref[1rP]{16S rDNA Purification}
\end{enumerate}

\section{16S rDNA Purification}\label{1rP}
\subsection{Overview}
This section describes how to purify the 16S rDNA amplicons from \hyperref[1rA]{16S rDNA Amplification} using the QIAGEN QIAquick PCR Purification Kit.
If you are using a different purification kit,
please follow its instructions and continue to \hyperref[SS]{Sanger Sequencing}.

\subsection{Materials}
\begin{itemize}
  \item 16S rDNA PCR amplicon samples
  \item QIAquick PCR Purification Kit
\end{itemize}

\subsection{Method}
Follow the subsequent instructions in parallel for all PCR amplicons synthesized in \hyperref[1rA]{16S rDNA Amplification}.
\begin{enumerate}
  \item Add 5 volumes of buffer PB to 1 volume of PCR sample and mix by pipet.
  \item Place a QIAquick spin column in a provided 2-mL collection tube. Then apply the sample to the column and centrifuge at $\geq17,900$ x $g$ for 60 s. Discard the flow-through.
  \item Add 750 $\mu$L of buffer PE to the column and centrifuge at $\geq17,900$ x $g$ for 60 s. Discard the flow-through.
  \item Centrifuge the column at $\geq17,900$ x $g$ for 60 s, then transfer the column to a clean 1.5-mL microcentrifuge tube.
  \item To elute, add 30 $\mu$L of buffer EB to the center of the column. Let stand for 1 min 60 s, and then centrifuge at $\geq17,900$ x $g$ for 60 s.
  \item Check the concentration of the purified product on the nano spectrophotometer. Then either store at $-20^\circ$C or proceed to \hyperref[SS]{Sanger Sequencing}.
\end{enumerate}

\section{Sanger Sequencing}\label{SS}
\subsection{Overview}
This section describes how to prepare the purified 16S rDNA samples obtained from \hyperref[1rP]{16S rDNA Purification} for Sanger sequencing by \href{https://www.genewiz.com/}{Genewiz}.
If you are using a different sequencing platform or provider,
please follow the relevant instructions,
as they most certainly will differ from what is listed here,
and continue to \hyperref[SAwC]{Species Abundances with CASEU}

\subsection{Materials}
\begin{itemize}
  \item Purified PCR 16S rDNA amplicons
  \item 27F forward primer
  \item 8-tube PCR strip
\end{itemize}
\subsection{Method}
Follow the subsequent instructions in parallel for all PCR amplicons purified in \hyperref[1rA]{16S rDNA Purification}.
\begin{enumerate}
  \item Dilute an aliquot of the 16S rDNA amplicon to a concentration of 3 ng\slash $\mu$L.
  \item Add the amplicon dilution, 27F forward primer, and ddH$_2$O to a PCR tube following the recipe in \hyperref[aB]{Appendix B}.
  \item Label tubes using the tube number with the prefix \emph{HL} (e.g. HL1, HL2, \ldots HL\textit{N}), noting what each tube contains.
  \item Submit a Genewiz Sanger sequencing order online and print the order form. Place both the PCR tubes and order form in a clear, sealable plastic bag and deposit in the Genewiz drop-box.
\end{enumerate}

\section{Species Abundances with CASEU}\label{SAwC}
\subsection{Overview}
This section describes how to quantify species abundances with the R package \textbf{CASEU}
(\textbf{C}ommunity \textbf{A}nalysis via \textbf{S}anger \textbf{E}lectropherogram \textbf{U}nmixing)
\cite{Cermak_2020}
using the sequencing results obtained from \hyperref[SS]{Sanger Sequencing}.
If you choose to use different software to quantify abundances,
I do not know why you are reading this protocol.

\subsection{Materials}
\begin{itemize}
  \item .ab1 sequencing files for pure and mixed cultures
  \item Computer with an installation of \href{https://www.r-project.org}{R}
\end{itemize}

\subsection{Method}
\begin{enumerate}
  \item Refer to the vignette found on the \href{https://htmlpreview.github.io/?https://bitbucket.org/dattamanoshi/caseu/raw/master/doc/CASEU_Vignette.html}{CASEU Repository} for instructions on installation and a full introduction to CASEU. Also, you can see \href{https://github.com/jpsco5890/quantify_16S_ratios.git}{Jack's Repository} for an example script.
\end{enumerate}

\appendix
\section{PCR}\label{aA}
\subsection{Master Mix Recipe}
Below is a recipe for a single,
40-$\mu$L 16S rDNA PCR using \href{https://www.neb.com/protocols/0001/01/01/pcr-protocol-m0530}{Phusion Polymerase}.
To generate a master mix for $N$ samples,
simply multiply the volumes by $N*1.1$ to account for pipetting errors.
\begin{table}[h]
  \begin{tabular}{|r|c|}
    \hline
    \textbf{Component} & \textbf{Volume ($\mu$L)} \\
    \hline
    \texttt{ddH$_2$O} & \texttt{20.8} \\
    \hline
    \texttt{5X HF reaction buffer} & \texttt{ 8.0} \\
    \hline
    \texttt{10mM dNTPs} & \texttt{ 0.8}\\
    \hline
    \texttt{3 $\mu$M 27F forward primer} & \texttt{ 4.0} \\
    \hline
    \texttt{3 $\mu$M 1492R reverse primer} & \texttt{ 4.0} \\
    \hline
    \texttt{Phusion polymerase} & \texttt{ 0.4} \\
    \hline
    \texttt{template gDNA} & \texttt{ 2.0} \\
    \hline
    \texttt{Total} & \texttt{40.0} \\
    \hline
  \end{tabular}
  \label{tab:PCR_MM}
\end{table}

\subsection{Thermocycler Program}
The following program can be found under Kapil's folder in the Hwa-lab thermocycler
(\texttt{NSB2106\textbackslash kapil\textbackslash 16s phusion}).
\begin{table}[h]
  \begin{tabular}{|r|c|c|}
    \hline
    \textbf{Step} & \textbf{Temperature ($^\circ$C)} & \textbf{Time (mm:ss)}\\
    \hline
    \texttt{1} & \texttt{98.0} & \texttt{00:30}\\
    \hline
    \texttt{2} & \texttt{98.0} & \texttt{00:10}\\
    \hline
    \texttt{3} & \texttt{54.0} & \texttt{00:30}\\
    \hline
    \texttt{4} & \texttt{72.0} & \texttt{01:30}\\
    \hline
    \multicolumn{3}{|c|}{\texttt{Steps 2-4 x24}}\\
    \hline
    \texttt{5} & \texttt{72.0} & \texttt{10:00}\\
    \hline
    \texttt{6} & \texttt{10.0} & \texttt{INFINITE}\\
    \hline
  \end{tabular}
  \label{tab:PCR_prog}
\end{table}

\section{Genewiz}\label{aB}
\subsection{Sample Mix Recipe}
Below is a recipe for a single,
15-$\mu$L 16S rDNA Sanger Sequencing sample when sequencing through \href{https://www.genewiz.com/}{Genewiz}.
\begin{table}[h]
  \begin{tabular}{|r|c|}
    \hline
    \textbf{Component} & \textbf{Volume ($\mu$L)} \\
    \hline
    \texttt{ddH$_2$O} & \texttt{11.0} \\
    \hline
    \texttt{3 $\mu$M 27F forward primer} & \texttt{ 2.0} \\
    \hline
    \texttt{15 ng\slash $\mu$L 16S rDNA amplicon} & \texttt{ 2.0} \\
    \hline
    \texttt{Total} & \texttt{15.0} \\
    \hline
  \end{tabular}
  \label{tab:SS_recp}
\end{table}

\bibliography{../../library-hwa_research}
\bibliographystyle{ieeetr}

\end{document}
