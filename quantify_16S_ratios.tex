\documentclass[twocolumn]{article}
\usepackage{cite}
\usepackage[colorlinks=true]{hyperref}

\begin{document}

\title{Protocol: Quantifying species abundance using 16S Sanger Sequencing and CASEU}
\author{Jack Reddan}

\maketitle{}

\tableofcontents{}

\section{Culture Collection}\label{CC}
\subsection{Overview}
This section describes collecting cultures for which you wish to quantify species abundances.
In addition to any experimental mixtures,
pure cultures of community members are needed for the analysis.

\subsection{Materials}
\begin{itemize}
  \item Pure and mixed liquid cultures
  \item 2-mL microcentrifuge tubes
\end{itemize}

\subsection{Method}
\begin{enumerate}
  \item In a 2-mL microcentrifuge tube, Collect 1 ODmL of liquid culture for all mixtures and pure cultures. Spin-down and remove supernatant.
  \item Either store pellets at $-20^\circ$C, or continue to \hyperref[GI]{gDNA Isolation}.
\end{enumerate}

\section{gDNA Isolation}\label{GI}
\subsection{Overview}
This section describes how to isolate gDNA to be used for 16S rDNA amplification.
These steps are for the QIAGEN DNeasy Blood \& Tissue Kit.
If you happen use a different kit,
please follow its instructions,
and then proceed to \hyperref[1rA]{16S rDNA Amplification}.

\subsection{Materials}
\begin{itemize}
  \item QIAGEN DNeasy Blood \& Tissue kit
  \item 100\% ethanol
  \item Cell pellets from cultures collected in \hyperref[CC]{Culture Collection}
\end{itemize}

\subsection{Method}
Follow the subsequent instruction in parallel for all samples to be processed.
\begin{enumerate}
  \item Resuspend pellet in 180 $\mu$L of buffer ATL, and add 20 $\mu$L of Proteinase K. Mix thoroughly by vortexing. Incubate at $56^\circ$C for 1 hour, and vortex every 15 minutes.
  \item Vortex for 15s, then add 200 $\mu$L of buffer AL and mix thoroughly by vortexing. Immediately add 200 $\mu$L of 100\% ethanol and, again, mix thoroughly by vortexing.
  \item Pipet the mixture from step 2 into a DNeasy Mini spin column placed in a 2-mL collection tube. Centrifuge at $\geq 6,000$ x $g$ for 1 min. Afterwards, discard flow-through and collection tube.
  \item Place DNeasy Mini spin column into a new 2-mL collection tube and add 500 $\mu$L of buffer AW1. Centrifuge at $\geq 6,000$ x $g$ for 1 min. Afterwards, discard flow-through and collection tube.
  \item Place DNeasy Mini spin column into a new 2-mL collection tube and add 500 $\mu$L of buffer AW2. Centrifuge at $20,000$ x $g$ for 3 min. Afterwards, discard flow-through and collection tube.
  \item Place DNeasy Mini spin column into a clean 1.5-mL microcentrifuge tube. Add 200 $\mu$L of buffer AE and incubate at room tempurature for 1 min. Centrifuge at $\geq 6,000$ x $g$ for 1 min to elute.
  \item Check gDNA concentration on the nano spectrophotometer. Then store at $-20^\circ$C or continue to \hyperref[1rA]{16S rDNA Amplification}.
\end{enumerate}

\section{16S rDNA Amplification}\label{1rA}
\subsection{Overview}
This section describes how to amplify the 16S rDNA using Phusion polymerase for Sanger sequencing.
If you are using a different polymerase,
please follow its specific instructions and proceed to \hyperref[SS]{Sanger Sequencing}.

\subsection{Materials}
\begin{itemize}
  \item Phusion polymerase
  \item 5X High Fidelity (HF) reaction buffer
  \item dNTPs
  \item 27F forward and 1492R reverse primers
  \item gDNA template from \hyperref[GI]{gDNA Isolation}
\end{itemize}

\subsection{Method}
Follow the subsequent instruction in parallel for all gDNA samples collected in \hyperref[GI]{gDNA Isolation}.
\begin{enumerate}
  \item Mix the ddH$_2$O, dNTPs, forward/reverse primers, 5X HF buffer, and polymerase according to \hyperref[aA]{Appendix A} for $N+1$ reactions, where $N$ is the number of gDNA samples to be processed.
  \item Pipet 38 $\mu$L of the PCR master mix into 200-$\mu$L PCR tubes. Then add 2 $\mu$L of template gDNA, at a concentration of 100 ng/$\mu$L, to the mix for each sample.
  \item Place reactions into the thermocycler and run the PCR program specified in \hyperref[aA]{Appendix A}.
  \item Once the run is complete, check for synthesis of the desired product using gel electrophoresis. If the proper product was amplified, either store the reaction at $-20^\circ$ C or proceed to \hyperref[1rP]{16S rDNA Purification}
\end{enumerate}

\section{16S rDNA Purification}\label{1rP}
\subsection{Overview}
This section describes how to purify the 16S rDNA amplicons from \hyperref[1rA]{16S rDNA Amplification} using the QIAGEN QIAquick PCR Purification Kit.
If you are using a different purification kit,
please follow its instructions and continue to \hyperref[SS]{Sanger Sequencing}.

\subsection{Materials}
\begin{itemize}
  \item 16S rDNA PCR amplicon samples
  \item QIAquick PCR Purification Kit
\end{itemize}

\subsection{Method}
Follow the subsequent instructions in parallel for all PCR amplicons synthesized in \hyperref[1rA]{16S rDNA Amplification}.
\begin{enumerate}
  \item Add 5 volumes of buffer PB to 1 volume of PCR sample and mix by pipet.
  \item Place a QIAquick spin column in a provided 2-mL collection tube. Then apply the sample to the column and centrifuge at $\geq17,900$ x $g$ for 60 s. Discard the flow-through.
  \item Add 750 $\mu$L of buffer PE to the column and centrifuge at $\geq17,900$ x $g$ for 60 s. Discard the flow-through.
  \item Centrifuge the column at $\geq17,900$ x $g$ for 60 s, then transfer the column to a clean 1.5-mL microcentrifuge tube.
  \item To elute, add 30 $\mu$L of buffer EB to the center of the column. Let stand for 1 min 60 s, and then centrifuge at $\geq17,900$ x $g$ for 60 s.
  \item Check the concentration of the purified product on the nano spectrophotometer. Then either store at $-20^\circ$C or proceed to \hyperref[SS]{Sanger Sequencing}.
\end{enumerate}

\section{Sanger Sequencing}\label{SS}
\subsection{Overview}

\subsection{Materials}

\subsection{Method}

\section{Quantify Species Abundances with CASEU}\label{QSAwC}
\subsection{Overview}

\subsection{Materials}

\subsection{Method}

\appendix
\section{PCR}\label{aA}
\subsection{Master Mix Recipe}

\subsection{Thermocycler Program}

\section{CASEU}\label{aB}
\subsection{Installation}

\subsection{Example Script}

\bibliography{../../library-hwa_research}
\bibliographystyle{ieeetr}

\end{document}
